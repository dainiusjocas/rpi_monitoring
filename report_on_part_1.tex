\documentclass[11pt, oneside]{article}   	% use "amsart" instead of "article" for AMSLaTeX format
\usepackage{geometry}                		% See geometry.pdf to learn the layout options. There are lots.
\geometry{letterpaper}                   		% ... or a4paper or a5paper or ... 
%\geometry{landscape}                		% Activate for for rotated page geometry
%\usepackage[parfill]{parskip}    		% Activate to begin paragraphs with an empty line rather than an indent
\usepackage{graphicx}				% Use pdf, png, jpg, or eps� with pdflatex; use eps in DVI mode
								% TeX will automatically convert eps --> pdf in pdflatex		
\usepackage{amssymb}
\usepackage{url}
\usepackage{hyperref}
\usepackage{glossaries}
\makeglossaries
\usepackage[xindy]{imakeidx}
%%\makeindex

%%----------------------------------------------------------------------------------------

\newglossaryentry{monitor}
{
  name=monitor,
  description={RPi which monitors agents. Also, this machine can be called  ''server`}
}

\newglossaryentry{agent}
{
  name=agent,
  description={RPi which is being monitored. Also, this machine can be called ''client``}
}

\newglossaryentry{snmp}
{
  name=SNMP,
  description={Simple Network Management Protocol \footnote{\url{http://en.wikipedia.org/wiki/Simple_Network_Management_Protocol}}}
}

\newglossaryentry{mib}
{
  name=MIB,
  description={Management Information Base\footnote{\url{http://en.wikipedia.org/wiki/Management_information_base}}}
}

%%--------------------------------------------------------------------------------------------

\title{RPI monitoring. Part 1}
\author{Dainius Jocas}
%\date{}							% Activate to display a given date or no date

\begin{document}
\maketitle
\section{Introduction}

This document describes how to deploy Raspberry Pi (RPi) cluster monitoring software.

\subsection{Scripts Used}

\begin{itemize}
  \item server\_jocas.py -- this script should be put on a \gls{monitor} (server).
  \item client\_1\_jocas.py -- to be deployed on the \gls{agent} (client).
  \item client\_2\_jocas.py -- to be deployed on the agent (client).
  \item DJ-MIB.py -- \gls{snmp} helper script. To be deployed on the agent (client).
  \item DJ-MIB -- custom \gls{mib} descriptor. To be deployed on the agent. 
\end{itemize}
\section{Deployment of scripts}

In this section guidelines (and not step by step instructions) will be provided. Note that no description how to get/generate actual code is given.

\subsection{Assumptions}

Agent knows address and the port of the monitor.

\subsection{On the monitor side}

So far monitor is super trivial -- all the information it receives script just forwards to the standard output. To run the server start server\_ jocas.py script.

\subsection{On the agent  side}

Steps to get project running:
\begin{enumerate}
  \item On the agent machine SNMP should be properly configured (lots of tricky work). 
  \item DJ-MIB should be deployed in one of the default directories for MIB. Check which are the directories can be done using command 
  \begin{verbatim}
  # net-snmp-config --default-mibdirs 
  \end{verbatim}
  \item Then remove snmpd program from startup programs. 
  \item Having root permissions startup client\_1\_jocas.py script.
  \item Start client\_2\_jocas.py script with proper parameters, e.g.
  \begin{verbatim}
  # python client_2_jocas.py [hostname] [port] [timeout in seconds]
  \end{verbatim}
\end{enumerate}

Thats it. If everything is OK monitor should receive reports on the state of the agent.  


\addcontentsline{toc}{subsection}{Glossary}
\printglossaries


\section{Part 2}

Here second part of the project will be described.

\subsection{Database}

To store data about RPis we decided to use SQLite database. It is because SQLite is very lightweight but at the same time it provides basic SQL functionality.

In the monitor install SQLite DBMS and python connector:
\begin{verbatim}
# apt-get install sqlite
# apt-get install python-sqlite
\end{verbatim}

Create database file:
\begin{verbatim}
# sqlite3 monitor.db
sqlite> .tables
\end{verbatim}
It will be created in the same directory.

Create a table rpi in the database run script create\_table.py:
\begin{verbatim}
# python create_table.py
\end{verbatim}

Columns of table rip are: \{ piID piHostname piActivity piCPULoad piGateway piStorageState piIP   piNetmask piProcesses piCluster piNetworkLoad  piDNS\}. Names of the columns are taken from DJ-MIB.

\subsection{Sequence to be up and running}

\begin{enumerate}
  \item run SNMP agent
  \item prepare file with list of monitors (put accurate IPs there)
  \item prepare databases in monitors
  \item make sure that the server is up and running
  \item start client\_strategy\_jocas script
  \item enjoy the show.
\end{enumerate}

\subsection{Some open issues}

\begin{enumerate}
  \item what about those bad fields like "cluster" because it is impossible to get them?
  \item How to define what is going on in the RPI?
  \item What would be optimal DB structure?
  \item where should DB be located?
  \item monitoring part? how to get data from all monitors?
\end{enumerate}


\end{document}  